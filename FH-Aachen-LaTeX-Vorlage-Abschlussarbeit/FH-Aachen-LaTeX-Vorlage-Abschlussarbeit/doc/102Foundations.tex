\clearpage
\chapter{\textbf{Foundations}}\label{grundlagen}
%\addtocontents{toc}{\vspace{0.8cm}}
\section{Setting}
The sewing machine in question is a Brother UF-8910.It is part of a textile production line in which a wristband is being produced. The wristband contains an RFID chip capable of storing diverse information. For instance, the technology could be utilized to store a personnel identification number. In the event that the technology is scanned at a workplace, the workplace's dimensions would adapt to conform to the worker's size. The production line has not been designed for implementation in actual industrial-scale manufacturing; rather, it functions as a model environment for workshop purposes. The sewing machine is utilized during the final stage of the wristband assembly process. The utilization of the machine is characterized by a straightforward procedural nature. The objective is to sew both ends of the open wristband together with a single seam to close it. The following figure illustrates this procedure.\\
\textbf{figure}
\\

The signals that can be extracted from the sewing machine encompass:
\begin{itemize}
	\item Second walking foot stroke: Two walking foot strokes can be set for different thikness of materials. This signal indicates if the walking foot is currently in the second stroke hight.
	\item Thread trimming: Indicates when the thread is trimmed
	\item Pressure foot: Indicates if the pressure foot is lifted
	\item Upper shaft rotating: Indicates if the machine is actively sewing
	\item Other than home screen: Indicates that the menu screen is currently not in the home screen
	\item Home screen and not sewing: Indicates that the menu screen is in the home screen and the machine is not actively sewing
\end{itemize}
The subsequent figure illustrates the various components of the sewing machine, thereby facilitating a more profound comprehension of the signals.\\
\textbf{figure}
\section{Definitions}
\subsection{Takt Time}
Maximum time allowed to produce one product in order to meet customer demand
\subsection{IoT and IIoT}
The term Internet of Things was first coined by \cite{ashtonThatInternetThings} when explaining the idea of combining RFID with the internet in an executive meeting. He explains that on the "normal" internet, most of the content is created by human beings. In contrast to this in the Internet of Things the data is generated by things and often describes things. But his emphasizes lays more on the description of things. For example to track and count them. The information to do so would come from sensors and RFID, he says.
Of course in these days more of the information on the internet is generated by bots and AI. But other than that the distinction still holds true. 
\\The Internet Society \cite{roseInternetThingsOverview} further explains that in the Internet of Things, machines are communicating with each other and are addressable via an own IP address. This standardizes the way in which devices communicate. They also mention that "Today, the Internet of Things has become a popular term for describing scenarios in which  Internet connectivity and computing capability extend to a variety of objects, devices, sensors, and everyday  items."
\\The Industrial Internet of Things is just the description of a domain where the IoT is used. In this case in manufacturing. \cite{WhatIoTInternet}
\section{State of the Art}\label{unterkapitel}
\subsection{Industrial IoT Architectures and Patterns}
Due to the requirement that the solution be developed utilizing IoT technologies and is set within a production context, a review of Industrial IoT (IIoT) architectures and patterns was conducted. The Industrial Internet Reference Architecture (IIRA) \cite{youngIndustrialInternetReference2022} serves as a comprehensive framework, offering valuable insights into various architectural models and design patterns relevant to this domain. This reference architecture describes the following patterns: IoT Component Capability Pattern, Three-Tier Architecture Pattern, Gateway-Mediated Edge Connectivity and Management architecture pattern, Digital Twin Core as a Middleware Architecture Pattern, Layered Databus Architecture Pattern, System-of-Systems Orchestrator Architecture Pattern. Of these patterns only the first two are applicable within the scope of this work. Therefore the other ones will only be described on the surface.
\paragraph{Architecture Patterns}
IoT architecture patterns define the structure and operation of various IoT systems, detailing their implementation and highlighting their unique characteristics.
\subparagraph{IoT Component Capability Model Pattern}
A single component and its associated capabilities are described, with the possibility that a component may comprise multiple sub-components. Consequently, the entire system can also be regarded as a component. The specific meanings of the capabilities are illustrated in the accompanying figure.
\begin{figure}[H]
	\includegraphics[width=\linewidth]{pic/IIRA-model-component-pattern.png}
	\caption{Component Capability Pattern. \\ (Young et al., 2022, S. 40)}
	\label{fig:Model-Component-Pattern}
\end{figure}
\subparagraph{Three-Tier Architecture Pattern}
The system comprises the Edge, Platform, and Enterprise Tiers, as well as connecting networks. The Edge Tier contains sensors and gateways that collect data. These are connected by the Proximity Network. Data preprocessing may already be happening there.
\\The Platform Tier is responsible for most data processing and storage via databases. It is connected to the Edge Tier via the Access Network.
\\The Enterprise Tier provides domain-specific applications and interfaces for end users. These are built upon the processed data from the platform tier. It also issues controls to lower tiers. This tier is connected to the Access Network via the Service Network.
The three tiers can also be further divided into different domains. That makes sense for bigger systems. But for a simple system as the one described in this work it is not necessary and therefore these domains will not be explained here.
\begin{figure}[H]
	\includegraphics[width=\linewidth]{pic/three-tier-architecture.jpg}
	\caption{Three Tier Architecture \\ (Young et al., 2022, S. 44)}
	\label{fig:Three-Tier-Architecture}
\end{figure}
\subsection{KPIs and Metrics for Performance Evaluation in Sewing Operations}
Performance measurement in the textile production industry is important due to intense competition. It enables producers to identify potential bottlenecks, provides a deeper understanding of processes, and facilitates more effective resource allocation \cite{alauddinProcessImprovementSewing2018}.
\\Key performance indicators (KPIs) and metrics serve as essential tools for performance measurement. Kang et al. \cite{kangHierarchicalStructureKey2016} from the National Institute of Standards and Technology in the United States analyzed the relationships among various types of KPIs and metrics used in operations management and production, based on the ISO 22400 standard.
Supporting elements, referred to as metrics in this thesis, describe the measured data necessary for calculating basic KPIs. These supporting elements are categorized into time and quantity.
Time elements quantify the duration of various events, such as the production time per unit. Conversely, quantity elements pertain to the number of produced items.
Maintenance elements capture information about machine-related issues. 
\\Based on the supporting elements, basic KPIs can be calculated. These KPIs are categorized into production, quality, and maintenance KPIs.
\\The researchers also emphasize the importance of comprehensive KPIs, which provide a broader overview of production performance. These KPIs build upon basic KPIs and include, for example, Overall Equipment Effectiveness (OEE), which is calculated by multiplying the KPIs for availability, performance, and quality ratio.
\\Other studies \cite{kironKPIKeyPerformance2022, alauddinProcessImprovementSewing2018} have specified KPIs specifically for the sewing section of a textile production plant. In this context, some KPIs overlap with those examined by Kang et al., while additional KPIs unique to the sewing section have also been introduced. The following table classifies these sewing-specific KPIs within the hierarchical framework proposed by Kang et al.
\begin{longtable}{|p{3cm}|p{7cm}|p{4cm}|}
	\caption{Classification of KPIs and Metrics in Sewing Section (based on Kang et al., 2016 and ISO 22400)} \\
	\hline
	\textbf{KPI/Metric} & \textbf{Description} & \textbf{Classification (Kang et al., 2016 / ISO 22400)} \\
	\hline
	\endfirsthead
	
	\hline
	\textbf{KPI/Metric} & \textbf{Description} & \textbf{Classification (Kang et al., 2016 / ISO 22400)} \\
	\hline
	\endhead
	% --- Supporting Element: Time ---
	\multicolumn{3}{|l|}{\textbf{Supporting Element: Time}} \\
	\hline
	Cycle Time & Total time taken to complete one operation, from start to start of the next piece. & Supporting Element: Time \\
	\hline
	Standard Minute Value (SMV) & Time required to complete a specific job under standard conditions and pace. & Supporting Element: Time \\
	\hline
	Allowance & Extra time permitted for personal needs, delays, and fatigue in production. & Supporting Element: Time \\
	\hline
	Idle Time/Machine & Time when operators or machines are not working, considered lost time. & Supporting Element: Time \\
	\hline
	% --- Supporting Element: Quantity ---
	\multicolumn{3}{|l|}{\textbf{Supporting Element: Quantity}} \\
	\hline
	Operation & A step in the process required to convert materials into a finished product. & Supporting Element: Quantity \\
	\hline
	Manpower to Machine Ratio & Ratio of workers to machines, used to optimize labor and production. & Supporting Element: Quantity \\
	\hline
	Absenteeism & Rate of operator absence, which affects production and efficiency. & Supporting Element: Quantity \\
	\hline
	No of Style Change & Frequency of style changes, impacting productivity, efficiency, and quality. & Supporting Element: Quantity \\
	\hline
	% --- Basic KPI: Production ---
	\multicolumn{3}{|l|}{\textbf{Basic KPI: Production}} \\
	\hline
	Efficiency & Comparison of actual output to what could be achieved with the same resources. & Basic KPI: Production \\
	\hline
	Productivity & Achievement toward goals based on the relationship between inputs and outputs. & Basic KPI: Production \\
	\hline
	Availability & Percentage of scheduled time employees or machines are productive. & Basic KPI: Production \\
	\hline
	Performance & Amount of product delivered relative to available productive time. & Basic KPI: Production \\
	\hline
	Line Wise Sewing Efficiency & Efficiency of sewing lines, often linked to man-to-machine ratio. & Basic KPI: Production \\
	\hline
	% --- Basic KPI: Quality ---
	\multicolumn{3}{|l|}{\textbf{Basic KPI: Quality}} \\
	\hline
	Defect per Hundred Units (DHU) & Number of defects found per hundred units produced. & Basic KPI: Quality \\
	\hline
	Quality & Percentage of perfect or saleable products produced. & Basic KPI: Quality \\
	\hline
	% --- Comprehensive KPI ---
	\multicolumn{3}{|l|}{\textbf{Comprehensive KPI}} \\
	\hline
	Overall Labor Effectiveness (OLE) & Measures workforce utilization, performance, and quality, reflecting labor's impact on productivity. & Comprehensive KPI \\
	\hline
	Overall Equipment Effectiveness (OEE) & Quantifies how well equipment performs relative to its designed capacity, considering availability, performance, and quality. & Comprehensive KPI \\
	\hline
\end{longtable}
\subsection{IoT-Plattforms}
The IoT is known for producing large amounts of data and for the potentials to grow these amounts even more. Therefore a scalable software infrastructure that is needed. That is where IoT-Plattforms come into play \cite{turkiEvaluatingOpenSource2024}. The authors also mention that IoT-Platforms help accelerating the solution development "[...] by providing foundational capabilities, avoiding the need to implement low-level infrastructure."
\\\cite{asemaniUnderstandingIoTPlatforms2019} further highlight the different capabilities that are typical for IoT-Platforms.
\paragraph{Connectivity and Device Management}
Through various communication protocols the platforms connect with the devices, enabling them to communicate with each other, manage device status and configurations, handle software updates and provide mechanisms for error reporting.
\paragraph{Data Storage, Management, Analysis, Visualization}
Through connections to databases they store large volumes of data often in the cloud or locally. Also further data processing and analytics through various methods as well as visualizations through dashboards are possible.
\paragraph{Development and Deployment Tools}
By providing APIs and SDKs the developers are enabled to further create custom applications.
\paragraph{Auditing and Payments}
The Platforms help to have an overview over the data or compute usage and the resulting costs.
\paragraph{Service Management}
By giving an oversight over parameters like resource consumption, data requirements and access, the user can monitor vertical as well as platform internal services. The platforms also enable the communication between services or combination of basic services to create new ones.
\paragraph{Integration}
Platforms can be integrated with each other, other data sources and the cloud. 
\paragraph{Fog/Edge Computing}
IoT-Platforms often support distributed data processing and storage. This can lead to less traffic due to processing close to the data source. Faster transmission would be enabled therefore and reinforced due to  shorter communication distances.
\\\\The researchers go on to reveal that while commercial platforms carry all of the mentioned capabilities, open source platforms are often focused on specific capabilities. Thus in implementation sometimes need to be combined to deliver a holistic IoT-Platform. 
\subsection{Differences between Relational and Timeseries Databases}
In the paper written by \cite{turkogluComparisonTimeSeries2024} it is analyzed how relational databases and time series databases compare regarding speed and storage efficiency when used in Grafana.
First they point out the use case for relational databases is for single time data points, which can be related to other data points over various tables. Therefore enabling complex queries involving joins, aggregations and multiple tables. They make sure the data is accurate and stays consistent.
Time series data bases on the other hand are made for data points with a timestamp and large volumes of data. This makes them ideal for IoT applications and real time data analytics. They are optimized to enable high speed read and write operations as well as efficient data storage.
The differences in query return time are already there with small amounts of data, but when scaling up the amount they become clearly visible as can be seen in the figure below.
\begin{figure}[H]
	\includegraphics[width=\linewidth]{pic/query-performance-db.png}
	\caption{Query Performance Relational vs. Time Series DB \\ (Turkoglu et. al, 2024, S. 2)}
	\label{fig:query-performance-db}
\end{figure}
This is only one of three test cases but they all give a similar picture that shows the superiority of time series databases regarding query return times.


\section{Legal Framework}
Personal rights regarding personal data are highly important in Germany and the entire European Union. Therefore, it is important to provide an overview of the regulations that the proposed system must fulfill.
According to the General Data Protection Regulation \cite{REGULATIONEU2016} of the European Union, it is illegal to automatically measure a worker's performance without human intervention that produces legal effects. Though this is not the intention of the system, if the produced data were traceable to a single worker, it could be abused to put pressure on the worker or find reasons to fire them if their performance is too low. The regulation also stipulates that workers must be informed if any personal data is being collected. In the case of the system in question, personal data could be produced in the form of amounts traceable to individual workers.
Additionally, according to the German "Betriebsverfassungsgesetz § 87 Mitbestimmungsrechte," \cite{87BetrVGEinzelnorm} the works council has co-determination rights regarding systems that measure worker performance.


