\clearpage
\chapter{\textbf{Kapitel 3}}\label{kap3}
\addtocontents{toc}{\vspace{0.8cm}}

\begin{table}[htb]
\caption{Messergebnisse}
\label{tab:messung}
\centering
\begin{tabu}{l|[2pt]C|C|C}
Stellung & \frac{T_U}{^\circ C}  & \frac{T_c}{^\circ C} & \frac{\Delta T}{^\circ C}  \\
\tabucline[2pt]{-}
senkrecht (0°) & 27,3 & 69,8 & 42,5\\
\tabucline[0.5pt]{-}
waagerecht (90°) & 26,6 & 70,6 & 44,0\\
\end{tabu}
\end{table}

\begin{table}[h]
\centering
\caption{Smartphone Sensordaten}
\begin{tabu}{|p{9cm}|l|l|}
\hline
Sensorinformation&Format&frequency [$s^{-1}$]\\
\hline
App identifier for vendor & int64 & once per transfer\\
WIFI and network carrier IP addresses& int128 & once per transfer\\
battery level& int8 & 0.1\\
Position information: latitude, longitude, altitude, speed, course, vertical position accuracy, horizontal position accuracy, floor level information& float32[8] & 1\\
Heading information: heading.x, heading.y, heading.z, true heading, magnetic heading, heading accuracy& float16[6] & 1 \\
Accelerometer information: acceleration.x, acceleration.y, acceleration.z& float16[3] & 2 \\
Gyroscope information: rotationRate.x, rotationRate.y, rotationRate.z& float16[3] & 2 \\
altimeter information: relative altitude, pressure & float16[2] & 1 \\ 
timestamp & uint32 & once per transfer \\
Temperature [°C] & float16 & 1\\
\hline
\end{tabu}
\label{tab:smartphonesensor}
\end{table}

Wie in Tabelle \ref{tab:smartphonesensor} zu sehen ist, ist es besser, Trennlinien nur dort einzusetzen, wo logische Grenzen liegen.