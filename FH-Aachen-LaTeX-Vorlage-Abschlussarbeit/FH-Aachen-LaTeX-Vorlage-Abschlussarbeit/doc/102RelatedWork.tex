\clearpage
\chapter{\textbf{Related Work}}\label{relatedWork}
This section presents an overview of related solutions for sewing machines, with a focus on methods for data collection and processing in monitoring applications. It also examines general approaches to performance measurement in manufacturing. The analysis will likely address the types of metrics underlying these solutions, the sensor technologies used for data acquisition, and the strategies for data processing, storage, and analytics. This review aims to clarify the current state of the art, identify possible gaps in existing systems, and position this work within the field of automated performance measurement systems. Furthermore, the review seeks to highlight potential limitations and opportunities, including technical aspects such as hardware, software, and integration, as well as architectural considerations like standards, frameworks, and open-source approaches, which may influence the implementation of performance measurement systems.
\\\\
In their seminal work, Jung et al. \cite{jungRemoteSensingSewing2020} propose a system for the analysis of sewing machine operator skill level and the complexity of the assigned sewing tasks. The present study bears a notable similarity to the aforementioned work, as both analyze the cycle time per unit. This objective is pursued by employing power consumption data of the sewing machines as the fundamental metric. To that end, a power monitoring system is employed, which is connected to the machines power plug without necessitating modifications to the sewing machine itself. Subsequently, the data is transmitted to a cloud server via wireless communication. The determination of quantity and time of work is achieved through the implementation of pattern analysis algorithms. The efficacy of the system is predicated on the non-intrusive nature of PMS. A notable disadvantage of this solution is the necessity of sufficient sample data to ensure the accuracy of the results.\\
Another system that uses power consumption measurement was proposed by Strazinskas \cite{strazinskasDevelopmentSolutionMonitoring2025}. The utilization of current sensors enabled the discernment of fluctuations in motor current, which are inherently associated with the operational states of the sewing machine, including initiation, cessation, seam length, and stitch speed. The sensor is connected to an Arduino Uno microcontroller, which transmits the received data to a Raspberry Pi 3B+ microcomputer. The microcomputer is responsible for central data collection and storage. The data is stored in files, which is a disadvantage of the system because it is less efficient than storing it in an optimized time series database. Therefore, a greater quantity of space is required, and the rate of data processing is reduced. Additionally, the absence of any preprocessing mechanisms underscores the necessity for optimized data management, a crucial aspect that demands significant memory resources. Strazinskas's research indicates that the utilization of the present sensor engenders measurement errors and noise in the data.\\
Although adopting a divergent approach, the system proposed by Quoc et al. \cite{quocApplyingIoTOperations2025} likewise relies on IoT devices (which were not further described) connected to the machine. The data of these devices is then transmitted to the cloud, where it is integrated with the enterprise's Management Information System (MIS). This integration facilitates the optimization of resource allocation. In light of the aforementioned data, a visualization is generated to illustrate the open positions of a contract and the extent to which it has been completed.\\
A divergent approach is posited by Wedanage et al. \cite{wedanageFogAssistedIndustrial2022} in their study "Fog Assisted Industrial Sewing," in which they utilize a mobile application to record the cycle times of the sewing step.To initiate and conclude the sewing process on the workpiece, the operator is required to engage a designated button at the commencement and cessation of the operation. The data is structured in a JSON format and published via MQTT under a designated topic. A fog node, implemented using a Cisco IR829 industrial router, is subscribed to the specified topic and processes the data through a Java application. Additionally, the application utilizes a RethinkDB database, which is characterized as open-source and specifically engineered for real-time applications. The fog node is already capable of performing local data analytics. In the cloud, data from all fog nodes is combined to provide real-time alerts and analytics through dashboards. These dashboards visualize cumulative average cycle times per member and compare them to takt time (maximum time allowed to produce one product to meet customer demand). The employment of fog computing facilitates the system's notable scalability, a consequence of the distribution of computing tasks across multiple fog nodes. This configuration facilitates the system's capacity for real-time data analytics and reduced bandwidth utilization. A notable drawback of this system is its reliance on manual input from operators, which introduces a margin of error. Moreover, no historical analytics have been conducted, despite the existence of a foundation for such analyses.
A notable drawback of the system in question, as well as those proposed by Jung et al. and Quoc et al., is the utilization of a cloud, which gives rise to several concerns, including but not limited to data security, latency, and ongoing costs.
The merits of the aforementioned systems include their capacity for retrofitting older machines and their non-intrusive nature.
\\\\
Beyond these closely related implementations, a broader perspective is provided by Tambare et al. \cite{tambarePerformanceMeasurementSystem2022}, who review various approaches to performance measurement systems. Initially, the discourse centers on two pivotal standards. The ISA-95 standard delineates entities at the shop floor level, where information technology systems such as ERP, CRM, cloud platforms, and SQL databases interact with operational technology components like sensors, actuators, microcontrollers, SCADA systems, and PLCs. Its primary function is to formalize production processes. Conversely, the ISO 22400 is employed for the formalization of performance metrics. The provided framework facilitates the definition, calculation, and dissemination of key performance indicators (KPIs), encompassing their contextual framework, formula, unit of measurement, range, and intended audience. The researchers then proceed to underscore the Scania case study, wherein these two standards are being utilized, as a paradigm of the practical implementation of international standards for performance measurement in a smart manufacturing context.\\
This case study by Samir et al. (samirKeyPerformanceIndicators2018) will now be examined in more detail. The system is composed of numerous autonomous, self-contained computational units that are each capable of performing independent computations and collaborating with each other. The software utilized for the processing and distribution of data has been developed using a Service Oriented Architecture (SOA) framework. This approach involves the implementation of multiple services that function independently of one another, with each service designated to specific responsibilities. The communication between the units is facilitated by an Enterprise Service Bus (ESB). The architecture further employs an Event-Driven Architecture (EDA), whereby events serve as triggers for services. This aims to address the issue of communication delays. The KPIs are designed in accordance with the ISO 22400 standard, and the IS-95 is employed to enable automated interfaces between enterprise and control systems.
\\\\
Taken together, these studies illustrate a variety of approaches to IoT-based machine monitoring, each with specific strengths and limitations.
In contrast to the systems examined in the aforementioned studies, the system presented in this thesis is designed to operate with a contemporary sewing machine that facilitates data extraction directly from the machine. It is also intended to provide a more comprehensive overview of the production process by means of various Key Performance Indicators (KPIs), which will be addressed in subsequent sections.
The system proposed in this thesis aims to incorporate the strengths of the discussed systems while mitigating their weaknesses. The system has been designed to align more closely with the proposed model in the case study by implementing standards and service-oriented software. Therefore, this facilitates enhanced scalability and maintainability.

