\chapter{\textbf{Requirements Analysis}}\label{grundlagen}
\subsection{Functional Requirements} % (from your requirements table)
\subsection{Non-Functional Requirements} %(scalability, security, real-time processing)
\subsection{Constraints} %(no additional sensors, existing infrastructure)
\subsection{KPI Selection and Justification}
The team around \cite{kangHierarchicalStructureKey2016} from the National Institute of Standards and Technologie in the U.S. has worked out the different kinds of KPI's that are being used in operation management and production and how the various metrics and KPI's are related to each other.
\\First there are the supporting elements which in this thesis will be called metrics because they describe the measured data that is needed to calculate the basic KPI's. The supporting elements are devided into the categories time and quantity.
\\Then there are maintenance elements which give insight into upkeep of the machines. Of which the following were able to be calculated with the given metrics.
\\