\chapter{\textbf{Requirements Analysis}}\label{grundlagen}
In order to select the most suitable technologies and design the system, it is first necessary to establish clear requirements. In this chapter, the aforementioned requirements will be enumerated and elucidated.
In order to maintain consistency regarding the compliance level, the terms "must," "should," and "will" were employed.
The following words shall be explained in brief. The term "must" is employed to signify an unconditional obligation, implying that the fulfillment of this requirement is not subject to discussion or negotiation. The term "should" conveys a degree of desirability, indicating that fulfillment of the requirement would be advantageous. The term "will" signifies that this particular requirement is currently under consideration for inclusion in the subsequent release. However, it is imperative to maintain awareness of this requirement so that the system can be designed in a manner that facilitates its seamless integration in the future.
\subsection{Functional System Requirements} % (from your requirements table)
\begin{tabularx}{\textwidth}{|X|X|}
	\hline
\textbf{Requirement}	& \textbf{Explanation} \\
	\hline
The system must show KPIs that are relevant for the sewing process	&  In the workshops there must be some KPIs that fit into the story of a textile production with a sewing process\\
	\hline
The system should show aditional KPIs that are relevant for the manufacturing industry in general	&  Workshop participants are from all sorts of companies within the manufacturing industry\\
	\hline
 The system must present these KPIs in a visual manner that provides information about the classification of the current value e.g. with colors and thresholds	& So that the management can act quickly upon the KPIs and does not need to lookup thresholds \\
	\hline
The system must provide the user with the ability to change the timeframe on which the KPIs are calculated	&  Especially when looking at historic data it is usefull to be able to set the timeframe\\
	\hline
The system should show graphs with historical data & Enables management to see trends and patterns\\
	\hline
\end{tabularx}

\subsection{Non-Functional Requirements} %(scalability, security, real-time processing)
\begin{tabularx}{\textwidth}{|X|X|}
	\hline
\textbf{Requirement}	&  \textbf{Explanation}\\
	\hline
The system must make use of open source software where possible	&  To be replecable by small and medium companies\\
	\hline
The system must be capable to generate all of the KPIs from the machine-data without installing any additional sensors	&  \\
	\hline
The system must be designed in a way that makes it easily scalable	&  Usually more than just one machine need to be considered for monitoring\\
	\hline
The system must be deployable with minimal effort	&  To be able to deploy elsewhere with not much effort \\
	\hline
The system must be capable to retrieve raw data via opc-ua	&  The PLC to which the sewing machine is connected publishes the data over opc-ua\\
	\hline
The system should make use of existing patterns, frameworks and solutions were possible	&  The system shall serve as a reference for other systems that are more readily implementable\\
	\hline
The system should update the KPIs in real-time (<10s)	&  To support timely interventions\\
	\hline
 The system must be able to run on a local machine and therefore independent of any cloud service	&  \\
	\hline
The system must provide the user with the ability to access the dashboard from within the shopfloor network	&  \\
	\hline
\end{tabularx}

\subsection{Constraints} %(no additional sensors, existing infrastructure)
\subsection{KPI Selection and Justification}
 Of which the following were able to be calculated with the given metrics.
\\